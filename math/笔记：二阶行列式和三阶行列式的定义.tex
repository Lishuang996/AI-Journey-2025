\documentclass[UTF8, 12pt]{article} % 使用12pt字体大小以避免小数点后的问题
\usepackage{ctex}
\usepackage{amsmath}
\usepackage{lmodern} % 使用Latin Modern字体,它有更好的字体大小支持
\usepackage{unicode-math} % 如果需要更好的数学符号支持

\title{笔记:二阶行列式和三阶行列式的定义}
\author{李爽}
\date{\today}

\begin{document}

\maketitle

\section{定义}
有以下一个方程组:
\begin{equation*}
\left\{
    \begin{aligned}
        &3x + 4y = 5 \\
        &7x + 9y = 11 \\
    \end{aligned}
\right.
\end{equation*}


解出来的是:
$$
x=\frac{5\times9-4\times11}{3\times9-4\times7}
$$

$$
y=\frac{3\times11-5\times7}{3\times9-4\times7}
$$


观察可知,$x$和$y$的解都可以表示为两个数相乘减两个数相乘。
为了表示方便,引出了另一个符号,即二阶行列式。


$$
\begin{vmatrix}
    a & b \\
    c & d
\end{vmatrix}
=a \times d - b \times c
$$


因此,$x$和$y$的解都可以表示为:
$$
x=\frac{\begin{vmatrix}
    5 & 4 \\
    11 & 9
\end{vmatrix}}{\begin{vmatrix}
    3 & 4 \\
    7 & 9
\end{vmatrix}}
$$
$$
y=\frac{\begin{vmatrix}
    3 & 5 \\
    7 & 11
\end{vmatrix}}{\begin{vmatrix}
    3 & 4 \\
    7 & 9
\end{vmatrix}}
$$


\section{克莱姆法则:快速求解二元一次方程}
现有以下一个方程组:
$$
\left\{
    \begin{aligned}
    &a_1 x + b_1 y=c_1\\
    &a_2 x + b_2 y=c_2
    \end{aligned}
\right.
$$

其中,$a_1,b_1,c_1,a_2,b_2,c_2$都是常数。


$$
x=\frac{
    \begin{vmatrix}
        c_1 & b_1 \\
        c_2 & b_2
    \end{vmatrix}
}{
    \begin{vmatrix}
        a_1 & b_1 \\
        a_2 & b_2
    \end{vmatrix}
}
\indent
y=\frac{
    \begin{vmatrix}
        a_1 & c_1 \\
        a_2 & c_2
    \end{vmatrix}
}{
    \begin{vmatrix}
        a_1 & b_1 \\
        a_2 & b_2
    \end{vmatrix}
}
$$


\section{二阶行列式的定义}
$$
\begin{vmatrix}
    a_{11}&a_{12}\\
    a_{21}&a_{22}
\end{vmatrix}
=a_{11}a_{22}-a_{12}a_{21}
$$

$a_{ij}$被称为一个\textbf{元素}。
$i$是\textbf{行},$j$是\textbf{列}。
从行列式的左上角到右下角被称为\textbf{主对角线},
右上角到左下角被称为\textbf{副对角线}。
对角线方向相乘并相减被称为\textbf{对角线法则}。

\section{三阶行列式的定义}
三阶行列式可表示为:
$$
\begin{vmatrix}
    a_{11}&a_{12}&a_{13}\\
    a_{21}&a_{22}&a_{23}\\
    a_{31}&a_{32}&a_{33}
\end{vmatrix}
$$

三阶行列式计算方法为:

原式$=a_{11}a_{22}a_{33}+a_{12}a_{23}a_{31}+a_{13}a_{21}a_{32}-a_{13}a_{22}a_{31}-a_{11}a_{23}a_{32}-a_{12}a_{21}a_{33}$

\subsection{上三角行列式}
如下所示的行列式就被称为上三角行列式。空白区域都是0。
$$
\begin{vmatrix}
    a_{11}&a_{12}&a_{13}\\
          &a_{22}&a_{23}\\
          &      &a_{33}
\end{vmatrix}
$$

此时原式$=a_{11}a_{22}a_{33}$

\subsection{下三角行列式}
下三角行列式如下所示,空白区域都是0。
$$
\begin{vmatrix}
    a_{11}&        &        \\
    a_{21}&a_{22}  &        \\
    a_{31}&a_{32}  &a_{33}
\end{vmatrix}
$$

同上三角行列式,此时原式$=a_{11}a_{22}a_{33}$

\subsection{主对角线行列式}
主对角线行列式如下所示,空白区域都是0。
$$
\begin{vmatrix}
    a_{11}&        &        \\
          &a_{22}  &        \\
          &        &a_{33}
\end{vmatrix}
$$

同上三角行列式,此时原式$=a_{11}a_{22}a_{33}$
\end{document}