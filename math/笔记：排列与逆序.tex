\documentclass[UTF8, 12pt]{article} % 使用12pt字体大小以避免小数点后的问题
\usepackage{ctex}
\usepackage{amsmath}
\usepackage{lmodern} % 使用Latin Modern字体,它有更好的字体大小支持
\usepackage{unicode-math} % 如果需要更好的数学符号支持
\usepackage{geometry}
\title{笔记:排列与逆序}
\author{李爽}
\date{\today}
\geometry{a4paper,left=2cm,right=2cm,top=0cm,bottom=0cm}
\begin{document}

\maketitle
\section{定义}
$n$级排列:由$1,2,\cdots,n$组成的一个有序数组。
例如:$123$,$321$,$213$,$231$,$312$,$132$,就被成为3级排列。

$1,2,\cdots,n$的排列有$n!$个。

$12\cdots n$被称为$n$级自然排列。

逆序数:一个排列中逆序对个数。记为$N(i_1 i_2 \cdots i_n)$或$\tau (i_1 i_2 \cdots i_n)$

奇排列:逆序数是奇数的排列。

偶排列:逆序数是偶数的排列。

对换:将其中的两个数进行交换。
每做一次对换,奇偶性改变一次。

$n$级排列一共有$n!$个。
奇排列和偶排列各占一半,分别为$\frac{n!}{2}$

\end{document}